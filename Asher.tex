\documentclass[11pt]{article}

\usepackage{mathtools}

\newcommand{\remove}[1]{}
\newcommand{\ignore}[1]{}
\newenvironment{ignoreme}{\ignore{}{}}



\usepackage{amsthm}


\usepackage{amsmath,amssymb,booktabs}
\usepackage{url}
\usepackage{cite}
\usepackage{color}

\newcommand{\authnote}[3]{\textcolor{#3}{[{\footnotesize {\bf #1:} { {#2}}}]}}
%\newcommand{\rnote}[1]{\authnote{R}{#1}{blue}}

\definecolor{DarkBlue}{RGB}{0,0,150}
\usepackage[colorlinks,linkcolor=DarkBlue,citecolor=DarkBlue]{hyperref}
%\usepackage[colorlinks,linkcolor=navy,citecolor=NavyBlue]{hyperref}

%%%%%%%%%%%%  Defining theorem-like environments %%%%%%%%%%
\newtheorem{theorem}{Theorem}[section]
\newtheorem{proposition}[theorem]{Proposition}
\newtheorem{definition}{Definition}[section]
\newtheorem{claim}[theorem]{Claim}
\newtheorem{lemma}[theorem]{Lemma}
\newtheorem{conjecture}{Conjecture}
\newtheorem{notation}[definition]{Notation}
\newtheorem{remark}{Remark}[section]
\newtheorem{corollary}[theorem]{Corollary}


%%%%%%%%%%%%%%%%% SOME BASIC DEFINITIONS %%%%%%%%%%%%%%%%%%

%Security Parameter
\newcommand{\secparam}{\lambda}
\newcommand{\secp}{\secparam}

% Vectors, Matrices and such

\def\veca{\vc{a}}
\def\vecb{\vc{b}}
\def\vecc{\vc{c}}
\def\vecd{\vc{d}}
\def\vece{\vc{e}}
\def\vecm{\vc{m}}
\def\vecp{\vc{p}}
\def\vecs{\vc{s}}
\def\vect{\vc{t}}
\def\vecv{\vc{v}}
\def\vecw{\vc{w}}
\def\vecx{\vc{x}}
\def\vecy{\vc{y}}

\def\Z{\mathbb{Z}}
\def\N{\mathbb{N}}
\def\R{\mathbb{R}}
\def\Q{\mathbb{Q}}
\def\F{\mathsf{F}}

% Emphasis in theorems
\newcommand{\emphth}[1]{{\rm\sf {#1}}}

% Changing QED symbol in claim proofs
\newenvironment{claimproof}{\begin{proof}
\renewcommand{\qedsymbol}{{$\blacksquare$}}
}{\end{proof}}


% Fixing bug in citations inside of brackets.
\newcommand{\brafix}[1]{[{#1}]}
\newcommand{\citefix}[2]{{\cite[#1]{#2}}}


%%%%%%%%%%%%%%%%%%%%%%%%%%%%%%%%%%%%%%%%%%%%%%%%%%%%%%%%%%%%

% Calligraphic and blackboard type letters.

\def\cA{{\cal A}}
\def\cB{{\cal B}}
\def\cC{{\cal C}}
\def\cD{{\cal D}}
\def\cE{{\cal E}}
\def\cF{{\cal F}}
\def\cG{{\cal G}}
\def\cH{{\cal H}}
\def\cI{{\cal I}}
\def\cJ{{\cal J}}
\def\cK{{\cal K}}
\def\cL{{\cal L}}
\def\cM{{\cal M}}
\def\cN{{\cal N}}
\def\cO{{\cal O}}
\def\cP{{\cal P}}
\def\cQ{{\cal Q}}
\def\cR{{\cal R}}
\def\cS{{\cal S}}
\def\cT{{\cal T}}
\def\cU{{\cal U}}
\def\cV{{\cal V}}
\def\cW{{\cal W}}
\def\cX{{\cal X}}
\def\cY{{\cal Y}}
\def\cZ{{\cal Z}}
%%%%%%%%%%%%%%%%%
\def\bbC{{\mathbb C}}
\def\bbE{{\mathbb E}}
\def\bbF{{\mathbb F}}
\def\bbG{{\mathbb G}}
\def\bbM{{\mathbb M}}
\def\bbN{{\mathbb N}}
\def\bbQ{{\mathbb Q}}
\def\bbR{{\mathbb R}}
\def\bbV{{\mathbb V}}
\def\bbZ{{\mathbb Z}}

\def\Zq{\bbZ_q}

\def\algA{{\mathsf{A}}}
\def\algE{{\mathsf{E}}}
\def\algP{{\mathsf{P}}}
\def\algM{{\mathsf{M}}}
\def\algS{{\mathsf{S}}}
\def\algV{{\mathsf{V}}}

\def\IP{{\left<\algP,\algV\right>}}


%%%%%%%%%%%%%%%%%


% Rounding commands

\newcommand{\ceil}[1]{\left\lceil #1 \right\rceil}
\newcommand{\floor}[1]{\left\lfloor #1 \right\rfloor}
\newcommand{\round}[1]{\left\lfloor #1 \right\rceil}


% Other short-hands

\def\binset{\{0,1\}}
\def\pmset{\{\pm 1\}}
\def\ind{\mathbbm{1}}
%\def\ind{\mathbf{1}}

\newcommand{\abs}[1]{\left\vert {#1} \right\vert}
\newcommand{\norm}[1]{\left\| {#1} \right\|}
\newcommand{\norminf}[1]{\left\| {#1} \right\|_{\infty}}


% Assignments
\def\getsr{\stackrel{\scriptscriptstyle{\$}}{\gets}}
\def\getsd{{:=}}
%\def\bydef{\stackrel{.}{=}}
\def\bydef{\triangleq}
\def\getsf{{\gets}}



% Asymptotics

\def\poly{{\rm poly}}
\def\polylog{{\rm polylog}}
\def\loglog{{\rm loglog}}
\def\polyloglog{{\rm polyloglog}}
\def\negl{{\rm negl}}
\newcommand{\ppt}{\mbox{{\sc ppt}}}
\def\Otilde{\widetilde{O}}

% Indistinguishability
\newcommand{\cind}{{\ \stackrel{c}{\approx}\ }}
\newcommand{\compind}{\cind}
\newcommand{\sind}{{\ \stackrel{s}{\approx}\ }}

%\newcommand{\sind}[1][\epsilon]{{\ \stackrel{{#1}}{\approx}_{s}\ }}
%\newcommand{\epssind}{{\ \stackrel{{\epsilon}}{\approx}_{s}\ }}


% Complexity classes

\def\NP{\mathbf{NP}}
\def\Ppoly{{\mathbf{P}/\poly}}


% Cryptographic assumptions

\newcommand{\ddh}{\mathrm{DDH}}
\newcommand{\cdh}{\mathrm{CDH}}
\newcommand{\dlin}{\text{\rm $d$LIN}}
\newcommand{\lin}{\text{\rm Lin}}
\newcommand{\sxdh}{\mathrm{SXDH}}
\newcommand{\rsa}{\mathrm{RSA}}
\newcommand{\sis}{\mathrm{SIS}}
\newcommand{\odsis}{\text{\rm 1D-SIS}}
\newcommand{\rodsis}{\text{\rm 1D-SIS-R}}
\def\odsisr{\rodsis}
\newcommand{\isis}{\mathrm{ISIS}}
\newcommand{\lwe}{\mathrm{LWE}}
\newcommand{\dlwe}{\mathrm{DLWE}}
\newcommand{\qr}{\mathrm{QR}}

%\newcommand{\gapsvp}{\mathrm{GapSVP}}
%\newcommand{\sivp}{\mathrm{SIVP}}

\def\bbZ{\mathbb{Z}}
\def\gapsvp{\mathsf{GapSVP}}
\def\gapSVP{\gapsvp}
\def\gapsivp{\mathsf{GapSIVP}}
\def\SIVP{\mathsf{SIVP}}
\def\sivp{\SIVP}

% Types of attacks

\newcommand{\adv}{\mathrm{Adv}}
\newcommand{\simul}{\mathrm{Sim}}
\newcommand{\simulator}{\mathrm{Sim}}
\newcommand{\dst}{\mathrm{Dist}}
\newcommand{\leak}{\mathrm{Leak}}
\newcommand{\forge}{\mathrm{Forge}}
\newcommand{\col}{\mathrm{Col}}
\newcommand{\invt}{\mathrm{Inv}}
\newcommand{\cpa}{\text{\rm CPA}}
\newcommand{\kdm}{\mathrm{KDM}}
\newcommand{\kdi}{\mathrm{KDM}^{(1)}}
\newcommand{\kdmn}{\mathrm{KDM}^{(\usr)}}
\newcommand{\good}{\mathrm{GOOD}}
\newcommand{\legal}{\mathrm{L}}


% Linear algebra

\newcommand{\mx}[1]{\mathbf{{#1}}}
\newcommand{\vc}[1]{\mathbf{{#1}}}
\newcommand{\gvc}[1]{\bm{{#1}}}
%\newcommand{\vc}[1]{\gvc{{#1}}}

\newcommand{\zo}{\{0,1\}}

% Probability

\newcommand{\Ex}{\mathop{\bbE}}
\newcommand{\cov}{\mathop{\text{\rm Cov}}}

%\newcommand{\sd}{\mathop{\text{\tt dist}}}
\newcommand{\sd}{\mathop{\Delta}}



% Cryptographic elements

%\newcommand{\pub}{pk}
\newcommand{\aux}{\mathsf{aux}}

\newcommand{\sk}{ \mathsf{sk}}
\newcommand{\ek}{ \mathsf{ek}}
\newcommand{\dk}{ \mathsf{dk}}
\newcommand{\tk}{ \mathsf{rk}}
\newcommand{\rk}{\tk}
\newcommand{\msk}{ \mathsf{msk}}
\newcommand{\pk}{ \mathsf{pk}}
\newcommand{\mpk}{ \mathsf{mpk}}
\newcommand{\vk}{ \mathsf{vk}}
\newcommand{\pp}{ \mathsf{pp}}
\newcommand{\out}{{\rm{out}}}
\newcommand{\inp}{{\rm{in}}}
\newcommand{\start}{{\rm{start}}}
\newcommand{\accept}{{\rm{accept}}}
\newcommand{\tgt}{{\rm{tgt}}}

\def\PK{\mathsf{PK}}
\def\MPK{\mathsf{MPK}}
\def\MSK{\mathsf{MSK}}
\def\SK{\mathsf{SK}}
\def\bydef{:=}

% Functinonal encryption keys
\newcommand{\fempk}{\MPK}
\newcommand{\femsk}{\MSK}


\newcommand{\fhepk}{\PK}
\newcommand{\fhesk}{\SK}


\def\CT{\mathsf{CT}}

% IBE keys
\newcommand{\ibempk}{{MPK_{ibe}}}
\newcommand{\ibemsk}{{MSK_{ibe}}}
\newcommand{\ID}{{ID}}
\newcommand{\id}{{ID}}
\newcommand{\pind}{\mathsf{ind}}
\newcommand{\halfq}{\lfloor q/2 \rfloor}
\newcommand{\nand}{\;\textsc{nand}\;}
\newcommand{\band}{\;\textsc{and}\;}
\newcommand{\bor}{\;\textsc{or}\;}

\newcommand{\crs}{\text{\sf crs}}

\newcommand{\params}{\mathsf{params}}
\newcommand{\state}{{setupstate}}

\newcommand{\query}{{query}}
\newcommand{\qstate}{{qstate}}
\newcommand{\resp}{{resp}}

% Protocols
\newcommand{\ibe}{\mathrm{IBE}}
\newcommand{\fhe}{\mathrm{FHE}}
\newcommand{\gfe}{\mathrm{GFE}}
\newcommand{\pke}{\mathrm{PKE}}
\newcommand{\atoe}{\mathrm{TOR}}
\newcommand{\tor}{\mathrm{TOR}}
\newcommand{\awtoe}{\mathrm{wTOR}}
\newcommand{\ator}{\mathrm{wTOR}}
\newcommand{\fe}{\mathrm{FE}}
\newcommand{\mpc}{\mathrm{MPC}}

% Algorithms


\newcommand{\keygen}{\mathsf{Keygen}}
\newcommand{\encode}{\mathsf{Encode}}
\newcommand{\recodegen}{\mathsf{ReKeyGen}}
\newcommand{\simrecodegen}{\mathsf{SimReKeyGen}}
\newcommand{\recode}{\mathsf{Recode}}
\newcommand{\decode}{\mathsf{Decrypt}}

\newcommand{\dmax}{d_{\mathrm{max}}}

\newcommand{\gen}{\mathsf{Gen}}
\newcommand{\eval}{\mathsf{Eval}}
\newcommand{\Eval}{\eval}
\newcommand{\setup}{\mathsf{Params}}
%\newcommand{\extract}{\mathsf{Extract}}
\newcommand{\enc}{\mathsf{Enc}}
\newcommand{\dec}{\mathsf{Dec}}
\newcommand{\eqct}{\mathsf{Equivocate}}

\newcommand{\symname}{\ssym}
\newcommand{\symkeygen}{\mathsf{SYM.Setup}}
\newcommand{\symgen}{\symkeygen}
\newcommand{\symsetup}{\symkeygen}
\newcommand{\symenc}{\mathsf{SYM.Enc}}
\newcommand{\symdec}{\mathsf{SYM.Dec}}

\newcommand{\pkesetup}{\mathsf{PKE.Setup}}
\newcommand{\pkeenc}{\mathsf{PKE.Enc}}
\newcommand{\pkedec}{\mathsf{PKE.Dec}}

\newcommand{\shname}{\mathsf{SH}}
\newcommand{\shkeygen}{\mathsf{SH.Keygen}}
\newcommand{\shgen}{\shkeygen}
\newcommand{\shenc}{\mathsf{SH.Enc}}
\newcommand{\shdec}{\mathsf{SH.Dec}}
\newcommand{\sheval}{\mathsf{SH.Eval}}

\newcommand{\hename}{\mathsf{HE}}
\newcommand{\hekeygen}{\mathsf{HE.Keygen}}

\newcommand{\hegen}{\hekeygen}
\newcommand{\heenc}{\mathsf{HE.Enc}}
\newcommand{\hedec}{\mathsf{HE.Dec}}
\newcommand{\heeval}{\mathsf{HE.Eval}}
\newcommand{\hesetup}{\mathsf{HE.Setup}}


\newcommand{\fhname}{\mathsf{FH}}
\newcommand{\fhkeygen}{\mathsf{FH.Keygen}}
\newcommand{\fhenc}{\mathsf{FH.Enc}}
\newcommand{\fhdec}{\mathsf{FH.Dec}}
\newcommand{\fheval}{\mathsf{FH.Eval}}

\newcommand{\fhename}{\mathsf{FHE}}
\newcommand{\fhekeygen}{\mathsf{FHE.Keygen}}
\newcommand{\fhegen}{\fhekeygen}
\newcommand{\fheenc}{\mathsf{FHE.Enc}}
\newcommand{\fhedec}{\mathsf{FHE.Dec}}
\newcommand{\fheeval}{\mathsf{FHE.Eval}}
\newcommand{\fhesetup}{\mathsf{FHE.Setup}}

\newcommand{\gswsetup}{\mathsf{SK.Setup}}
\newcommand{\gswgen}{\mathsf{SK.Keygen}}
\newcommand{\gswenc}{\mathsf{SK.Enc}}
\newcommand{\gswdec}{\mathsf{SK.Dec}}
\newcommand{\gsweval}{\mathsf{SK.Eval}}

\newcommand{\mwname}{\mathsf{SHMK}}
\newcommand{\mwenc}{\mathsf{SHMK.Enc}}
\newcommand{\mwdec}{\mathsf{SHMK.Dec}}
\newcommand{\mweval}{\mathsf{SHMK.Eval}}
\newcommand{\mwext}{\mathsf{SHMK.Extend}}
\newcommand{\mwexp}{\mathsf{SHMK.Expand}}

\newcommand{\mhmuname}{\mathsf{FDMK}}
\newcommand{\mhmusetup}{\mathsf{FDMK.Setup}}
\newcommand{\mhmugen}{\mathsf{FDMK.Keygen}}
\newcommand{\mhmuenc}{\mathsf{FDMK.Enc}}
\newcommand{\mhmudec}{\mathsf{FDMK.Dec}}
\newcommand{\mhmueval}{\mathsf{FDMK.Eval}}
\newcommand{\mhmunand}{\mathsf{FDMK.NAND}}

\newcommand{\lmkname}{\mathsf{LMK}}
\newcommand{\lmksetup}{\mathsf{LMK.Setup}}
\newcommand{\lmkgen}{\mathsf{LMK.Keygen}}
\newcommand{\lmkenc}{\mathsf{LMK.Enc}}
\newcommand{\lmkdec}{\mathsf{LMK.Dec}}

\newcommand{\gennext}{\mathsf{GenNext}}
\newcommand{\conv}{\mathsf{Conv}}

% Lattice Stuff

\newcommand{\trapgen}{\mathsf{TrapGen}}
\newcommand{\trapinvert}{\mathsf{TrapInv}}
\newcommand{\gpvsamp}{\mathsf{GPVSamp}}

\newcommand{\lat}{\Lambda}
\newcommand{\shiftedperplat}[2]{\lat^{\perp}_{{#1}+{#2}}}
\newcommand{\shiftedlat}[2]{\lat_{{#1}+{#2}}}

\newcommand{\vecpsi}{\boldsymbol{\psi}}
\newcommand{\vecmu}{\boldsymbol{\mu}}
\newcommand{\vecgamma}{\boldsymbol{\gamma}}
\newcommand{\vecr}{\mathbf{r}}
\newcommand{\vecu}{\mathbf{u}}
\newcommand{\vecg}{\mathbf{g}}


\newcommand{\GSnorm}{}

% Secret sharing
\newcommand{\sshare}{\mathsf{SS.Share}}
\newcommand{\srecover}{\mathsf{SS.Reconstruct}}

% Randomized encoding
\newcommand{\reencode}{\mathsf{RE.Encode}}
\newcommand{\redecode}{\mathsf{RE.Decode}}
\newcommand{\resim}{\mathsf{RE.Sim}}

\newcommand{\yaogen}{\mathsf{Yao.Gen}}
\newcommand{\yaoeval}{\mathsf{Yao.Eval}}
\newcommand{\yaoext}{\mathsf{Yao.Extract}}
\newcommand{\yaosim}{\mathsf{Yao.Sim}}


\newcommand{\randgen}{\mathsf{Rand.Gen}}
\newcommand{\calcpshare}{\mathsf{Calc.PShare}}

\newcommand{\ssgen}{\mathsf{SS.Gen}}
\newcommand{\vssgen}{\mathsf{VSS.Gen}}
\newcommand{\gcgen}{\mathsf{GC.Gen}}

\newcommand{\Setup}{\mathsf{Setup}}
\newcommand{\KeyGen}{\mathsf{KeyGen}}
\newcommand{\Enc}{\mathsf{Enc}}
\newcommand{\Dec}{\mathsf{Dec}}
\newcommand{\Sim}{\mathsf{Sim}}

\newcommand{\ibesetup}{\mathsf{IBE.Setup}}
\newcommand{\ibekeygen}{\mathsf{IBE.Keygen}}
\newcommand{\ibeenc}{\mathsf{IBE.Enc}}
\newcommand{\ibedec}{\mathsf{IBE.Dec}}
\newcommand{\ibepp}{\mathsf{IBE.PP}}


\newcommand{\fesetup}{\mathsf{FE.Setup}}
\newcommand{\fekeygen}{\mathsf{FE.Keygen}}
\newcommand{\feenc}{\mathsf{FE.Enc}}
\newcommand{\fedec}{\mathsf{FE.Dec}}

\newcommand{\pesetup}{\Setup}
\newcommand{\pekeygen}{\KeyGen}
\newcommand{\peenc}{\Enc}
\newcommand{\pedec}{\Dec}

\newcommand{\bdsetup}{\mathsf{BdFE.Setup}}
\newcommand{\bdkeygen}{\mathsf{BdFE.Keygen}}
\newcommand{\bdenc}{\mathsf{BdFE.Enc}}
\newcommand{\bddec}{\mathsf{BdFE.Dec}}
\newcommand{\bdsim}{\mathsf{BdFE.Sim}}

\newcommand{\bfsetup}{\mathsf{BFFE.Setup}}
\newcommand{\bfkeygen}{\mathsf{BFFE.Keygen}}
\newcommand{\bfenc}{\mathsf{BFFE.Enc}}
\newcommand{\bfdec}{\mathsf{BFFE.Dec}}
\newcommand{\bfsim}{\mathsf{BFFE.Sim}}


\newcommand{\oneqsetup}{\mathsf{OneQFE.Setup}}
\newcommand{\oneqkeygen}{\mathsf{OneQFE.Keygen}}
\newcommand{\oneqenc}{\mathsf{OneQFE.Enc}}
\newcommand{\oneqdec}{\mathsf{OneQFE.Dec}}
\newcommand{\oneqsim}{\mathsf{OneQFE.Sim}}



% Primitives
\newcommand{\prf}{\mathsf{PRF}}

\newcommand{\pe}{\mathcal{ABE}}
\newcommand{\pename}{\pe}
\newcommand{\ibename}{\mathcal{IBE}}
\newcommand{\ssym}{\mathcal{SYM}}
\newcommand{\pkename}{\mathcal{PKE}}
\newcommand{\ssname}{\mathcal{SS}}
\newcommand{\fename}{\mathcal{FE}}
\newcommand{\yaogs}{\mathcal{GS}}
\newcommand{\re}{\mathcal{RE}}
\newcommand{\bdfename}{\mathcal{BDFE}}
\newcommand{\oneqfename}{\mathcal{ONEQFE}} %??
\newcommand{\bffename}{\mathcal{BFFE}}

\newcommand{\nizk}{\text{\rm NIZK}}
\newcommand{\ltdf}{\text{\rm LTDF}}


% Tiny subscript

\newcommand{\tsub}[2]{{{#1}_{\scriptscriptstyle #2}}}


% "TH" symbol

\newcommand{\thh}{{\mbox{\tiny th}}}


% Bold paragraph (similar to \paragraph*{})

\newcommand{\boldpar}[1]{\vspace{3pt}\par\noindent\textbf{#1}}



\newcommand{\rnote}[1]{\authnote{R}{#1}{blue}}
\newcommand{\znote}[1]{\authnote{Z}{#1}{red}}



%%%%%%%%%%%%%%%%%%% Vnote  %%%%%%%%%%%%%%%%%%%%%%%%%%%%%

\newcounter{auxthmctr}
\newcounter{auxsecctr}

% Document specific definitions

\newcommand{\Adv}{\mathsf{Adv}}
\newcommand{\AdvPE}[1]{\mathsf{Adv}^{\textsc{pe}}_{#1}(\secp)}
\newcommand{\AdvCOW}[1]{\mathsf{Adv}^{\textsc{cp}}_{#1}(\secp)}
\newcommand{\idl}[1]{\left\langle{#1}\right\rangle}
\newcommand{\rlwe}{\mathsf{RLWE}}
\newcommand{\plwe}{\mathsf{PLWE}}
\newcommand{\drlwe}{\text{\rm G-RLWE}}
\newcommand{\sdrlwe}{\text{\rm RLWE}}
\newcommand{\vssm}{\text{\rm SVSS}}

\newcommand{\ekdm}{{\cE_{\kdm}}}
\newcommand{\usr}{{\nu}}

\newcommand{\add}{\mathsf{add}}
\newcommand{\mlt}{\mathsf{mult}}

\newcommand{\hc}{\hat{c}}
%\newcommand{\vsig}{\gvc{\sigma}}
%\newcommand{\hvsig}{\hat{\gvc{\sigma}}}

\newcommand{\mcyc}{{\Phi_{2^{\round{\log \secp}}}}}

\newcommand{\otild}{{\widetilde{O}}}
\newcommand{\omtild}{{\widetilde{\Omega}}}

\newcommand{\linf}{{\ell_{\infty}}}

\newcommand{\gf}{{\text{GF}}}

\newcommand{\zset}[1]{\{0, \ldots, {#1}\}}

\newcommand{\medskipo}{\medskip}
\newcommand{\medskipoo}{\medskipo}


\newcommand{\db}{{\mathtt{DB}}}
\newcommand{\dbs}{{|\db|}}

\newcommand{\esw}{\shname}
\newcommand{\esym}{\symname}

\newcommand{\hsk}{{hsk}}
\newcommand{\hpk}{{hpk}}
\newcommand{\hevk}{{hevk}}
\def\hesk{\hsk}
\def\hepk{\hpk}

\newcommand{\symsk}{{symsk}}


\newcommand{\hvcs}{\hat{\vc{s}}}
\newcommand{\hmxA}{\hat{\mx{A}}}
\newcommand{\hvce}{\hat{\vc{e}}}
\newcommand{\hvcb}{\hat{\vc{b}}}
\newcommand{\he}{{\hat{e}}}
\newcommand{\heta}{{\hat{\eta}}}
\newcommand{\hpsi}{{\hat{\psi}}}
\newcommand{\hPsi}{{\hat{\Psi}}}
\newcommand{\hchi}{{\hat{\chi}}}
\newcommand{\hvcv}{{\hat{\vc{v}}}}
\newcommand{\hw}{{\hat{w}}}
\newcommand{\hvca}{{\hat{\vc{a}}}}
\newcommand{\hb}{{\hat{b}}}
\newcommand{\hB}{{\hat{B}}}
\newcommand{\homega}{{\hat{\omega}}}



\newcommand{\pir}{\mathrm{PIR}}


\def\Hrand{H_{\mathsf{rand}}}
\def\polyclass{\mathsf{Poly}}
\def\colarith{\mathsf{Arith}}

%\newcommand{\tab}{\ \ \ }
\newcommand{\ttab}{\ \ \ \ \ \ }
\newcommand{\tttab}{\ \ \ \ \ \ \ \ \ }

\def\CT{\mathsf{ct}}
\def\ct{\mathsf{ct}}
\def\setI{\mathcal I}
\def\funct{F}
\def\cfam{\mathcal{C}}
\def\tfunct{\mathcal{G}}
\def\fn{F}
\def\tfn{G}


\def\redecode{\mathsf{RE.Decode}}
\def\reencode{\mathsf{RE.Encode}}
\def\resim{\mathsf{RE.Sim}}





%%%%%%%%%%%%%%%%%%%%%%%%%%%%%%%%%%%%%%%%%%%%%%%%%%%%%%%%%%%%%%%%%%%%%%
%%%%%%%%%%%%%%%%%%%%%%%%%%%%%%%%%%%%%%%%%%%%%%%%%%%%%%%%%%%%%%%%%%%%%%
%%%%%%%%%%%%%%%%%%%%%%%%%%%%%%%%%%%%%%%%%%%%%%%%%%%%%%%%%%%%%%%%%%%%%%


\newcommand{\matA}{\mathbf{A}}
\newcommand{\matB}{\mathbf{B}}
\newcommand{\matE}{\mathbf{E}}
\newcommand{\matM}{\mathbf{M}}
\newcommand{\matT}{\mathbf{T}}
\newcommand{\matR}{\mathbf{R}}
\def\matr{\matR}

\newcommand{\mA}{\matA}
\newcommand{\matU}{\mathbf{U}}
\newcommand{\matV}{\mathbf{V}}

\newcommand{\trapsamp}{\mathsf{TrapSamp}}
\newcommand{\eqnsolve}{\mathsf{EqnSolve}}


\newcommand{\absnewline}{\ifnum\full=0 \\ \fi}

\renewcommand{\cD}{\Psi}


\def\apf{{\cal APF}}
\def\tinyapf{{\scriptscriptstyle\cal APF}}
\def\apfcirc{{\cH^\tinyapf_{\secp,k}}}
\def\apfsize{{z^\tinyapf(\secp)}}
\def\apfdepth{{d^\tinyapf(\secp)}}

\def\pubp{{\cal PP}}


\def\vech{\vc{h}}
\def\Ginv{\mathbf{G}^{-1}}

\newcommand{\ginv}[1]{\mx{G}^{-1}({#1})}
\def\str{\vc{s}^T}
%\def\etr{\vc{e}^T}
%\def\etr{\mathsf{noise}}
\def\nse{\mathsf{noise}}
\def\noise{\nse}
\def\ma{\mx{A}}
\def\univ{\cU}


\newcommand{\psp}[2]{{#1}^{({#2})}}


\def\univckt{\univ}
\newcommand{\lwround}[2]{\round{#2}_{#1}}
\newcommand{\LWRound}[2]{\lwround{#1}{#2}}



\newcommand{\lwr}{\mathrm{LWR}}

\def\htt{{t'}}


\def\vecd{\mathbf{d}}


\def\Ginv{\mathbf{G}^{-1}}



\def\genpp{\mathsf{MatrixGen}}
\def\simpp{\mathsf{SimPP}}
%\def\matgen{\mathsf{MatrixCompute}}
\def\simmatgen{\mathsf{SimMatrixGen}}
%\def\ctgen{\mathsf{LWECompute}}
\def\computeR{\mathsf{ComputeR}}
\def\genmaster{\mathsf{GenMaster}}


\def\ctgen{\computeC}


\def\matB{\mathbf{B}}
\def\matC{\mathbf{C}}
\def\matD{\mathbf{D}}


\def\KeyGen{\mathsf{KeyGen}}
\def\Puncture{\mathsf{Puncture}}
\def\Constrain{\mathsf{Constrain}}
\def\EvalP{\mathsf{EvalP}}

\def\vecz{\mathbf{z}}
\def\matS{\mathbf{S}}
\def\vech{\mathbf{h}}

\def\matG{\mathbf{G}}

\def\CPRF{\mathcal{CPRF}}
\def\cprf{\CPRF}

\def\scprf{\mathcal{SCPRF}}



\def\OKeyGen{\mathsf{KeyGen}}
\def\OConstrain{\mathsf{Constrain}}
\def\OEval{\mathsf{Eval}}
\def\OConstrainEval{\mathsf{ConstrainEval}}
\def\KeyGen{\OKeyGen}
\def\Constrain{\OConstrain}
\def\Eval{\OEval}
\def\ConstrainEval{\OConstrainEval}


\def\matI{\mathbf{I}}
\def\vecst{\vecs^T}
\def\vecet{\vece^T}


\def\computeA{\mathsf{ComputeA}}
\def\computeC{\mathsf{ComputeC}}
\def\computeR{\mathsf{ComputeR}}



\def\glbmat{{\mx{\hat{D}}}}
\def\globmat{\glbmat}
\def\mata{\matA}
\def\matb{\matB}
\def\matc{\matC}
\def\matd{\matD}


\def\pspxi{\psp{x}{i}}
\def\pspwi{\psp{w}{i}}

\def\vcd{\vc{d}}
\def\vcdd{\vc{d}'}


% Experimental hybrid counter system

\newcounter{hybridcount}
\newcounter{prevhybridcount}
\newcounter{nexthybridcount}

\newcommand{\resethyb}{\setcounter{prevhybridcount}{-2} \setcounter{nexthybridcount}{0} \setcounter{hybridcount}{-1}}
\newcommand{\newhyb}{\stepcounter{prevhybridcount}\stepcounter{nexthybridcount}\refstepcounter{hybridcount}}
\newcommand{\thishyb}{\arabic{hybridcount}}
\newcommand{\prevhyb}{\arabic{prevhybridcount}}
\newcommand{\nexthyb}{\arabic{nexthybridcount}}

\def\hyb{\mathsf{H}}


% % % % % % % % % % %


\def\stb{\vc{u}}


\def\str{\vc{s}^T}
\def\nse{\mathsf{noise}}
\def\noise{\nse}
\def\ma{\mx{A}}
\def\mb{\mx{B}}
\def\mc{\mx{C}}
\def\mr{\mx{R}}
\def\mg{\mx{G}}
\def\mgi{\mx{G}^{-1}}
\def\mh{\mx{H}}



\newcommand{\mtd}[2]{\mx{{#1}}^{-1}_{{#2}}}
\newcommand{\mtdr}[1]{\mtd{{#1}}{\tau}}

\def\tda{\mtd{A}{\tau_0}}
\def\vmc{\vec{\mc}}
\def\vmb{\vec{\mb}}
\def\vmr{\vec{\mr}}
\def\vmg{\vec{\mg}}
\def\mhs{\mh^*}
\def\vmhs{\vec{\mh}^*}


\newcommand{\evk}{{evk}}
\newcommand{\msg}{{\mu}}

\newcommand{\ckt}{\Psi}
\newcommand{\prog}{\Pi}
\newcommand{\var}{\mathsf{var}}

\def\lq{\lceil \log q \rceil}
\def\ellq{{\ell_q}}

\hbadness=10000 \vbadness=10000

\setlength{\oddsidemargin}{.25in}
\setlength{\evensidemargin}{.25in} \setlength{\textwidth}{6in}
\setlength{\topmargin}{-0.4in} \setlength{\textheight}{8.5in}
\newcommand{\details}[1]{
   \renewcommand{\thepage}{\arabic{page}}
   \noindent
	\begin{center}
   \framebox{
      \vbox{
    \hbox to 5.78in {}
       \vspace{4mm}
       \hbox to 5.78in { {\Large \hfill {\bf Representation of P-adic Groups}  \hfill} }
       \vspace{2mm}
       \hbox to 5.78in { {\hfill {\it #1}} }
      }
   }
   \end{center}
   \vspace*{4mm}
}

\newcommand{\lecture}[1]{\details{#1}}
\usepackage{graphicx}
\usepackage{mathtools}	
\usepackage{algorithm}
\usepackage[noend]{algpseudocode}
\makeatletter
\def\BState{\State\hskip-\ALG@thistlm}
\makeatother

\newtheorem{thm}{Theorem}[section]
\newtheorem{cor}[thm]{Corollary}
\newtheorem{prop}[thm]{Proposition}
\newtheorem{lem}[thm]{Lemma}
\newtheorem{conj}[thm]{Conjecture}
\newtheorem{quest}[thm]{Question}

\theoremstyle{definition}
\theoremstyle{lemma}
\theoremstyle{conclusion}
\newtheorem{conclusion}[theorem]{Conclusion}
\newtheorem{question}[theorem]{Question}

\def\zint{\mathbb{Z}}
\def\zp{\zint_{p}}
\def\qp{\mathbb{Q}_{p}}

\def\zmodnz#1{\zint/#1\zint}
\def\zmodpz{\zmodnz p}
\def\zmodpmz#1{\zmodnz{p^{#1}}}
\def\zmodpnz{\zmodnz{p^{n}}}
\def\gl#1{GL(2, #1)}
\def\glzp{\gl{\zp}}
\def\glqp{\gl{\qp}}
\def\matr#1 #2 #3 #4{
\
  \begin{pmatrix}
    {#1} & {#2}  \\
    {#3} & {#4} 
  \end{pmatrix}
\
}


\makeatletter
\let\c@equation\c@thm
\makeatother
\numberwithin{equation}{section}


\begin{document}
\details{Noam Peri, 204686034}


I will solve questions 
from the book [GH]
\begin{quest} %TODO 6.2%
Define $red_n : \glzp \rightarrow \gl {\zmodpnz}$ by \\
$\matr a b c d \mapsto  \matr \overline{a} \overline{b} \overline{c} {\overline{d}}$ 
\\where $a \mapsto \overline{a}$ is the natural transformation "mod $p^n$"
\begin{enumerate}

\item prove that $red_n$ is well defined
\begin{proof}
We will use the following claim:
Let R be a commutative ring with a unit and $M \in M_{n \times n}(R)$. Then M is invertible iff det(M) is invertible in R.
Notice that in both $\zmodpnz$ and in $\zp$ an element x is invertible iff $p \nmid x$
\\Finally note that for any matrix M in either $\glzp$ or $\gl {\zmodpnz}$
\\ $\matr a b c d = \matr {a_0 + pa'} {b_0 + pb'} {c_0 + pc'} {d_0 + pd'}$ is invertible
$ \iff p \nmid det\matr {a_0 + pa'} {b_0 + pb'} {c_0 + pc'} {d_0 + pd'} \iff p \nmid [det \matr {a_0} {b_0} {c_0} {d_0} + p \cdot A] \iff p \nmid det(\matr {a_0} {b_0} {c_0} {d_0}) = det(red_1(M))$
\\ And thus we can conclude that M is invertible iff $red_1(M)$ is invertible iff $red_n(M)$ is invertible.
\end{proof}
\item prove that $red_n$ is onto $\gl {\zmodpnz}$ with kernel $K_n$
\begin{proof}
Denote by  $i : \zmodpnz \xhookrightarrow {} \zp$ the natural inclusion morphism. 
We can expand it to $i^*: \glzp \rightarrow \gl {\zmodpnz}$
\\ Then its clear that $red_n \circ i^* = id_{\gl {\zmodpnz}}$ and thus $red_n$ must be onto $\gl {\zmodpnz}$. Clearly, $$ker (red_n) = \{\matr a b c d | \overline{a}=1, \overline{b}=0, \overline{c}=0,  \overline{d}=1\} = \{\matr {1 + p^na'} {0 + p^nb'} {0 + p^nc'} {1 + p^nd'} | a',b',c',d' \in \zp \} = K_n$$
\end{proof}
\item Calculate the index $[\glzp:K_n]$
\begin{proof}
Using the last results and the  homomorphism theorem, we can see that:
\\$\glzp/K_n \cong \gl {\zmodpnz}$ 
\\ We will count the size of the latter expression. Every matrix $M \in \gl {\zmodpnz}$ can be written as $M = \matr a b c d + p \cdot \matr {a'} {b'} {c'} {d'}$ where $a,b,c,d \in [p-1]$ and $a',b',c',d' \in [p^{n-1}-1]$. Using the observations from the first part, M is invertible iff the first matrix is invertible in $\glzp$.
And thus $|\gl {\zmodpnz}| = p^{4(n-1)}\cdot |\gl {\zmodpz}|$.\\ 
It easy to calculate $|\gl {\zmodpz}| = (p^2 - 1)(p^2 - p)$ (Initially choosing a non zero vector, and then choosing another one that is not proportional to the first one). 
\\Concluding that $[\glzp:K_n] = p^{4(n-1)} (p^2 - 1)(p^2 - p)$
\end{proof}

\end{enumerate}
\end{quest}
\begin{quest} %TODO 6.3%
For a $n > m$ show that $K_n \unlhd K_m$ and calculate $|K_m/K_n|$.
\begin{proof}
From Definition $K_n \leqslant K_m \leqslant \glzp$. Using Question 6.2 $K_n \lhd \glzp $ (As the kernel of a group homomorphism). 
\\ And so $K_n \unlhd K_m \unlhd \glzp$. 
\\ Using the homomorphism theorems we know that: 
$$[\glzp : K_m]\cdot[K_m:K_n] = [\glzp:K_n]$$
Using question 6.2 result we get that:
$$[K_m:K_n] = \frac{[\glzp:K_n]}{[\glzp : K_m]} = \frac{p^{4(n-1)} (p^2 - 1)(p^2 - p)}{p^{4(m-1)} (p^2 - 1)(p^2 - p)} = p^{4(n-m)}$$
\end{proof}
\end{quest}
\begin{quest}%TODO 6.5%
Let $(\pi, V)$ be a representation of $\glqp$. Show that the following 2 definitions for smoothness are equivalent:
\\a. $\forall v \in V$ the function:
\\ $T_v: \gl \qp \rightarrow V$, defined by: $T_v(g) = \pi(g)v$, is locally constant.
\\b. $\forall v \in V \textbf{ } \exists n \in \mathbb{N}^+ \textbf{ }s.t \textbf{ }\forall k \in K_n \textbf{ }\pi(k)v=v$
\begin{proof}
a $\Rightarrow$ b:
\\ For every $v \in V$, we know that: $T_v(I_2) = \pi(I_2)v = v$
\\ Due to the fact that $T_v$ is locally constant, there is a neighborhood $U$ of $I_2$ such that:
$$\forall g \in U \textbf{ } T_v(g) = v$$
Using question 6.1, we know that $\{K_n\}_{n\geq0}$ is a basis for the open neighborhood of $I_2$, and thus there is $n \in \mathbb{N}$ such that $I_2 \in K_n \subseteq U $. Concluding that $\forall g \in K_n \textbf{ } T_v(g) = v$, as needed.
\\ b $\Rightarrow$ a:
\\ Let $g \in \glzp$, $v \in V$, there exists $n>0$ such that:
$$\forall k \in K_n \textbf{ } T_v(k) = v$$. 
\\ $U = g \cdot K_n$ is an open neighborhood of g(as multiplying by g is a open transformation).
\\ Every element $\varphi \in g \cdot K_n$ can be written as $\varphi = g \cdot k$ for some $k \in K_n$.
$$T_v(\varphi)=T_v(g \cdot k)=\pi(g\cdot k)v = \pi(g)[\pi(k)v]=\pi(g)v=T_v(g)$$, and  $T_v$ is constant on $U$.
\end{proof}
\end{quest}
\begin{quest}%TODO 6.6%
Let $(\pi, V)$ be a smooth representation of $\glqp$
\begin{enumerate}
\item Prove that $V^{K_m} \subseteq V^{K_n}$ for $m \leq n$, and that $\bigcup_{n \geq 1} V^{K_n} = V$
\begin{proof}
Let $m \leq n$ then $K_n \subseteq K_m$ as so:
$$ V^{K_m} = \{v \in V | \forall k \in K_m \textbf{ } \pi(k)v=v\} \subseteq \{v \in V | \forall k \in K_n \textbf{ } \pi(k)v=v\} = V^{K_n}$$
In addition using definition b (from question 6.5) of smoothness, for every $v \in V$ there is $n \geq 1$ such that $v \in V^{K_n}$. 
\\Concluding that $V = V^{K_n}$
\end{proof}


\item Show that $V^{K_n}$ is $\glzp$-invariant for each integer $n \geq1$
\begin{proof}
In question 6.2 we've seen that $K_n \lhd \glzp$, than for every $k \in K_n$ and $g \in \glzp$ there is a $k' \in K_n$ such that $kg = gk'$.
\\Let $v \in V^{K_n}, g \in \glzp$, we will show that $\pi(g)v \in V^{K_n}$. Indeed for every $k \in K_n$: 
$$ \pi(k)[\pi(g)v] = \pi(k \cdot g) v = \pi(g \cdot k')v = \pi(g) [\pi(k')v] = \pi(g)v$$ 
\end{proof}
\item Conclude that if $(\pi, V)$ is admissible, then every $v \in V$ lies in a finite-dimensional $\glzp$-invariant subspace of V.
\begin{proof}
The admissibility of $(\pi, V)$ means that $V^{K_n}$ is finite-dimensional for every n. From part 1 and 2 we know that they are $\glzp$-invariant and cover V. 
\end{proof}

\end{enumerate}
\end{quest}

\begin{quest}%TODO 6.22%
Recall that $\mu$ is a left invariant measure on the collection of compact open subsets of $\glzp$
\begin{enumerate}

\item Suppose $\delta = \matr {p^{e_1}} 0 0 {p^{e_2}} $, where $e_1 \leq e_2$ are integers. For each $n \geq 1$ prove the equality of indices:
$$[K_n:\delta^{-1}K_n\delta \cap K_n]=p^{e_2 - e_1} = [\delta^{-1}K_n\delta:\delta^{-1}K_n\delta \cap K_n]  $$
\begin{proof} First we will characterize the subgroup $\delta^{-1}K_n\delta \cap K_n$. 
\\ Let $M= \matr a b c d \in K_n$, then $\delta M \delta^{-1} = \matr {a} {p^{e_1 - e_2} b} {p^{e_2 - e_1} c} {d}$. Recall that $e_2 - e_1 \geq 0$, and thus the requirements from $a, d, p^{e2 - e1}c$ are  fulfilled.   
$$M \in \delta^{-1}K_n\delta \Longleftrightarrow \delta M \delta^{-1} \in  K_n \Longleftrightarrow p^n | p^{e1 - e2}b \Longleftrightarrow p^{n + e_2 - e_1} | b$$
We can now calculate the first index. For any 
$M = \matr a b c d, N = \matr A B C D \in K_n$ are in the same coset of $\delta^{-1}K_n\delta \cap K_n$ iff $MN^{-1} \in  \delta^{-1}K_n\delta \cap K_n$.
N is clearly invertible and we can write$N^{-1} = \frac{1}{|N|} \matr {D} {-B} {-C} {A}$
\\Due to the fact that $K_n$ is a subgroup we only need to check that the upper right element of the result $x = \frac{1}{N}(-aB + bA)$ satisfies  $p^{n + e_2 - e_1} |  x$ (in $\zp$).
$N \in K_n$ and so $|N| \equiv \frac{1}{|N|} \equiv 1(mod \ p\zp) $. We can than reduce the requirement to
$bA - aB \equiv  p^{n + e_2 - e_1}$ ,and again using the fact that $A, a \equiv 1(p)$ and thus invertible modulo $ p^{n + e_2 - e_1}$
\\ We can finally conclude that M, N are in the same coset iff $$\frac{b}{a} \equiv \frac{B}{A}\  mod \ ( p^{n + e_2 - e_1})$$
And so the coset is determined by this ratio.
Knowing that $a \equiv 1, b \equiv 0 ( p^n)$, we can conclude that $\frac{b}{a} \equiv 0 (p^n)$. And so the ratio is always part of the set $\{x \cdot p^n \mid x \in \{0, ..., p^{e_2- e_1} -1\}\}$. Choosing $a = 1, b = x\cdot p^n$ each element of this set is obtained. We can conclude that there are exactly $p^{e_2- e_1}$ cosets.
To calculate the second index will use the following observations:  $\delta^t = \delta,\ (\delta^{-1})^{t} = \delta^{-1}, \ M \in K_n \text{ iff } M^t \in K_n$.
\\ We define an automorphism of $\glqp$:  $\varphi: M \mapsto \delta^{-1} M^t  \delta$. 
\\From the previous observations we can verify that $\varphi$ maps $K_n$ onto $\delta^{-1} K_n  \delta$ and vice versa. And thus it induces an isomorphism between them and we conclude that
$$[K_n:\delta^{-1}K_n\delta \cap K_n]=
[\varphi(K_n):\varphi(\delta^{-1}K_n\delta \cap K_n)] = [\delta^{-1}K_n\delta:\delta^{-1}K_n\delta \cap K_n] = p^{e_2 - e_1} $$
\end{proof}
\item Suppose $g \in \glqp$. prove the equality of indices: 
\\ $[K_n:g^{-1}K_ng \cap K_n] =  [g^{-1}K_n g :g^{-1}K_n g \cap K_n]$
\begin{proof}
Using the Cartan decomposition any such $g \in \glqp$ can be written as $g = x \delta y$ with $x,y \in \glzp$ and $\delta = \delta(e_1,e_2)$ as in part 1.
We define an autorphism $\psi: \glqp \rightarrow \glqp$ by: $\psi: M \mapsto y M y^{-1}$.
\\ $\psi(K_n) = K_n$ (as $y \in \glzp$, $K_n\lhd \glzp)$
\\ $\psi(g^{-1}K_ng) =y \cdot( y^{-1}\delta^{-1}x^{-1}K_n x \delta y) \cdot y^{-1} = \delta^{-1}(x^{-1}K_nx)\delta = \delta^{-1}K_n\delta$
\\ Using those observations:
\\ $[K_n:g^{-1}K_ng \cap K_n]\stackrel{\psi}{=}[K_n:\delta^{-1}K_n\delta \cap K_n] \stackrel{1}{=}  [\delta^{-1}K_n\delta:\delta^{-1}K_n\delta \cap K_n] \stackrel{\psi^{-1}}{=} [g^{-1}K_n g :g^{-1}K_n g \cap K_n]$
\end{proof}
\item Let I be the common index from part (2) prove that:
$$ \mu(K_n) = I \cdot \mu(g^{-1}K_ng \cap K_n)= \mu(g^{-1}K_ng)$$ for all $g \in \glqp$ Deduce that $\mu$ is right invariant.
\begin{proof}
First notice that $g^{-1}K_ng$ in an compact open subset of $\glqp$ as multiplying by an element from wither side is an continuous and open morphism. An intersection of open compact subsets is also compact and open, and thus $\mu(g^{-1} K_n g)$, $\mu(g^{-1} K_n g \cap K_n)$ are well defined.
Using part 2, we know that both $K_n$ and $g^{-1} K_n g$  are a disjoint union of  I cosets of $ g^{-1} K_n g \cap K_n$ and thus $\mu(K_n) = I \mu( g^{-1} K_n g \cap K_n) = \mu( g^{-1} K_n g)$.
$\mu(K_n) = \mu(  g^{-1} K_n g) \stackrel{*}{=} \mu(g \cdot  g^{-1} K_n g) = \mu(K_n \cdot g)$ (we used the fact that $\mu$ is left invariant in *). And thus is right invariant.
\end{proof}
\end{enumerate}
\end{quest}
\end{document}
